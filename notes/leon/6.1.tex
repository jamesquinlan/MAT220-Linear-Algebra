%\section{Outline}
%\input{includes/thesis}
%Given a $m \times n$ system of equations: 
%\section*{Define arithmetic operations ($+, -, \times$) on matrices}
% \section*{Introduction}


\section*{Objectives}
\begin{enumerate}
	\item Define eigenvalues and eigenvectors
	\item Define characteristic polynomial
	\item Find Eigenvalues
	\item Find Eigenvectors
    	
\end{enumerate}






\rule[0.01in]{\textwidth}{0.0025in}
% ---------------------------------------------------- % 


\section{Eigenvalues}

\begin{definition}
	Let $A$ be an $n \times n$ matrix.  A scalar $\lambda$ is an \textbf{eigenvalue} of $A$ if there exists a  nonzero vector ${\bf x}$ such that $$A{\bf x} = \lambda {\bf x}.$$
	
	  The vector ${\bf x}$ is the \textbf{eigenvector} belonging to eigenvalue $\lambda$.
\end{definition}
 
 
 \begin{example}
 	Let 
	\[ {\bf x} = (2, 1)^T  \; \; \text{   and  }  \;\; A =  \begin{bmatrix} 4 & -2 \\ 1  & 1 \end{bmatrix}. \]
	
	Note that \[ A {\bf x} = 3 {\bf x} \]
	
	Therefore, $\lambda = 3$ is an eigenvalue and ${\bf x}$ is the associated eigenvector.
 \end{example}



\subsection{Finding Eigenvalues}
The insight of finding eigenvalues comes from considering the following:
\begin{align*}
A {\bf x} &= \lambda {\bf x} \\
A {\bf x} - \lambda {\bf x} &= {\bf 0} \\
(A  - \lambda I) {\bf x} &= {\bf 0} \\
\end{align*}
To obtain a nontrivial solution, $N(A - \lambda I) \ne {\bf 0}$.  That is, $A-\lambda I$ is singular.  Therefore, $\det (A - \lambda I) = 0$.  Let, 
\[  p(\lambda) =   \det (A - \lambda I)  \]


\subsection*{Steps to find eigenvalues}
\begin{enumerate}
	\item Find the determinant of $A - \lambda I$.  This will be a polynomial in $\lambda$, namely $ p(\lambda)$.
	\item Set $p(\lambda) = 0$ and solve for $\lambda$.  You could try factoring or using the quadratic formula  (if it is degree 2).  The solutions are the eigenvalues.  
\end{enumerate}	

\rule[0.01in]{\textwidth}{0.0025in}

\begin{example}
	\[ A =  \begin{bmatrix} 4 & -2 \\ 1  & 1 \end{bmatrix}. \]
	Using the ordered list above, we have, 
	
	\begin{enumerate}
	\item Find the determinant of $A - \lambda I$.  
	\begin{align*}   
	\begin{vmatrix} 4 -\lambda & -2 \\ 1  & 1-\lambda \end{vmatrix} &= (4-\lambda)(1-\lambda)-(-2)(1)\\
	&= 4-5 \lambda + \lambda^2 +2\\
	 &= \lambda^2 - 5\lambda+6\\
	 &= p(\lambda)
	 \end{align*}
	
	
	
		\item Set $p(\lambda) = 0$ and solve for $\lambda$.  
		\begin{align*}
			\lambda^2 - 5\lambda+6 &= 0\\
			(\lambda - 2)(\lambda - 3) &= 0\\
			\Rightarrow \lambda = 2 \;\; \text{ or } \;\; \lambda  = 3
		\end{align*}
		Therefore the eigenvalues of $A$ are 
		\[  \text{Eigenvalues } = \{ 2,3 \} \]
		\end{enumerate}	


	
\end{example}

\rule[0.01in]{\textwidth}{0.0025in}
In MATLAB
\begin{verbatim}
>> A = [4 -2; 1 1];
>> lambda = eigs(A);
\end{verbatim}
\rule[0.01in]{\textwidth}{0.0025in}

\begin{fact}
Geometric multiplicity is less than or equal to the algebraic multiplicity.
\end{fact}

\rule[0.01in]{\textwidth}{0.0025in}
% ---------------------------------------------------- % 






\section{Eigenvectors}
Associated with each eigenvalue are nonzero \textbf{eigenvector}(s).  
\subsection{Finding Eigenvalues}
\begin{enumerate}
\item For each eigenvalue, $\lambda_i$, in the spectrum, find 
\[ N(A - \lambda_i I) \]
by solving 
\[  (A - \lambda_i I) {\bf x} = {\bf 0} \]


\item Write as an \textbf{eigenspace}, (the span of the eigenvectors associated with the particular eigenvalue).  

\end{enumerate}



\subsection{Equivalent Statements}
\begin{enumerate}
	\item 
\end{enumerate}



\rule[0.01in]{\textwidth}{0.0025in}
% ---------------------------------------------------- % 



\begin{example}
Let \[ A =  \begin{bmatrix} 4 & -2 \\ 1  & 1 \end{bmatrix}. \]
Find the eigenvectors of $A$.  We have already found the eigenvalues, $\lambda_1 = 2$ and $\lambda_2=3$.  We have to find the eigenvectors associated with each eigenvalue.  Let's start with $\lambda_1 = 2$.

\[  (A - 2I) {\bf x} = {\bf 0} \]
Therefore, we solve
\[   
\begin{bmatrix} 4-2 & -2 & 0 \\ 1  & 1-2 & 0   \end{bmatrix}  \Rightarrow \begin{bmatrix} 2 & -2 & 0 \\ 1  & -1 & 0   \end{bmatrix}  \Rightarrow \begin{bmatrix} 2 & -2 & 0 \\ 0  & 0 & 0   \end{bmatrix}
\]
Therefore, we have that $x_2$ is a free variable, say $x_2 = \alpha$.  Then we backsubstitute to get that $x_1 = \alpha$.  Therefore, the eigenvector associated with $\lambda  = 2 $ is, 
\[  {\bf x} = \alpha  \begin{bmatrix} 1 \\ 1   \end{bmatrix} \]
The eigenspace  is span by $(1,1)^T$.



Next we find the eigenvector associated with $\lambda = 3$.  Similar to above
\[   
\begin{bmatrix} 4-3 & -2 & 0 \\ 1  & 1-3 & 0   \end{bmatrix}  \Rightarrow \begin{bmatrix} 1 & -2 & 0 \\ 1  & -2 & 0   \end{bmatrix}  \Rightarrow \begin{bmatrix} 1 & -2 & 0 \\ 0  & 0 & 0   \end{bmatrix}
\]
Therefore, we have that $x_2$ is a free variable, say $x_2 = \alpha$.  Then we backsubstitute to get that $x_1 = 2\alpha$.  Therefore, the eigenvector associated with $\lambda  = 3 $ is, 
\[  {\bf x} = \alpha  \begin{bmatrix} 1 \\ 2   \end{bmatrix} \]
The eigenspace  is span by $(1,2)^T$.

\end{example}

\rule[0.01in]{\textwidth}{0.0025in}
% ---------------------------------------------------- % 










\section*{Next time...}
Section 5.3: Least Squares Problem

