 


%\subsubsection*{Objectives}
%\begin{itemize}
%	\item Calculate length of a vector
%	\item Calculate dot products
%	\item Use dot product to determine  
%\end{itemize}



\subsubsection*{Learning outcomes}
Be able to:
\begin{itemize}
	 \item List and apply (algebraic) rules for matrix operations
	 \item Identify the identity matrix
	 \item Take powers of matrices
	 \item Be able to multiply matrices in various ways
	 \item Understand block matrices and block multiplication

	
\end{itemize}





\rule[0.01in]{\textwidth}{0.0025in}
% ---------------------------------------------------- % 


%
%%
%%%						  
%%%% SECTION:  Matrices
%%%
%%
%
\section*{Definitions, theorems, and examples}
\begin{enumerate}
\item  \textbf{Algebraic Rules}:  Let $A, B, C$ be matrices and $a, b \in \mathbb{R}$ .  The following are valid algebra rules:
\begin{enumerate}
	\item $A+B = B+A$
	\item $(A+B)+C = A+(B+C)$
	\item $(AB)C = A(BC)$
	\item $A(B+C) = AB + BC$
	
	\item $(A+B)C = AC + BC$
	\item $(ab)A = a(bA)$
	\item $ a (AB) = (aA)B = A(aB)$
	\item $(a+b)A = aA+bA$
	\item $a(A+B) = aA +aB$
\end{enumerate}




% EXAMPLE
\begin{example} If $a = 2$, and 
$$ A = \begin{bmatrix}  1& 2& 1\\ 3 & 3 & 5 \\ 2  &  4  &  1 \end{bmatrix} \text{\;\;\;\; and \;\;\;\;} B = \begin{bmatrix}  1& 0& 2\\ 2 & 1 &  1 \\ 5  &  4  &  1 \end{bmatrix}$$ 
then
 \begin{align*}
  2(A+B)  &= 2 \left( \begin{bmatrix}  1& 2& 1\\ 3 & 3 & 5 \\ 2  &  4  &  1 \end{bmatrix}+  \begin{bmatrix}  1& 0& 2\\ 2 & 1 &  1 \\ 5  &  4  &  1 \end{bmatrix} \right) \\
  &= 2 \begin{bmatrix} 2 & 2  &3 \\ 5 & 4 & 6 \\ 7  &  8  &  2  \end{bmatrix}\\
  &= \begin{bmatrix} 4 & 4  &6 \\ 10 & 8 & 12 \\ 14  &  16  &  4  \end{bmatrix}
 \end{align*}
 
 On the other hand, 
  \begin{align*}
  2A+2B  &=  2 \begin{bmatrix}  1& 2& 1\\ 3 & 3 & 5 \\ 2  &  4  &  1 \end{bmatrix}+ 2 \begin{bmatrix}  1& 0& 2\\ 2 & 1 &  1 \\ 5  &  4  &  1 \end{bmatrix}  \\
  &=   \begin{bmatrix}  2& 4& 2\\ 6 & 6 & 10 \\ 4  &  8  &  2 \end{bmatrix}+  \begin{bmatrix}  2& 0& 4\\ 4 & 2 &  2 \\ 10  &  8  &  2 \end{bmatrix} \\
  &= \begin{bmatrix} 4 & 4  &6 \\ 10 & 8 & 12 \\ 14  &  16  &  4  \end{bmatrix}
 \end{align*}
\end{example}




% Example 
\begin{example} If
$$ A = \begin{bmatrix}  1& 2& 3 & 4\\ 5 & 6 & 7  & 8 \end{bmatrix} \text{\;\;\;\; and \;\;\;\;} B = \begin{bmatrix}  3& 1& 5 & 2\\ -1 & 2 & 4  & 1 \end{bmatrix}$$ 
then
 $$ A+B = \begin{bmatrix}  4& 3& 8 & 6\\ 4 & 8 & 11  & 9 \end{bmatrix} = B+A$$ 


\end{example}

NOTE:  In general $AB \ne BA$

\rule[0.01in]{\textwidth}{0.0025in}
% ---------------------------------------------------- % 






\item \textbf{Power of a matrix (notation)}:  Repeated multiplication of an $n \times n$ matrix $A$ with itself:
\[  A^k = \underbrace{AA \cdots A}_{\text{$k$ times}} \]


\begin{example} $A^3 = AAA$
\end{example}


\rule[0.01in]{\textwidth}{0.0025in}
% ---------------------------------------------------- % 



% IDENTITY MATRIX
\item \textbf{Identity matrix}:  A special matrix that acts like the identity element $1 \in \mathbb{R}$.  That is, the matrix $I$ such that $AI = A$ and $IA=A$ is called the identity matrix for any $n \times n$ matrix $A$.

\begin{definition}[Identity matrix]
The $n \times n$ \textbf{identity matrix} is $I = (\delta_{ij})$,  where

\[ 
 \delta_{ij} =  \begin{cases} 	1 & \text{ if \;} i = j\\
						0 & \text{ if \;} i \ne j
			\end{cases}
			\]
\end{definition}



\textbf{Alternative notation}:  The $n \times n$ identity matrix can be written using the row of columns notation, 
$$ I  = ({\bf e}_1 \;\; \;{\bf e}_2 \; \;\; {\bf e}_3 \; \;\; \cdots \; \;\; {\bf e}_n)$$
The $j$-th column of $I$ is $${\bf e}_j = \begin{bmatrix} 0 \\ \vdots \\ 1 \\ \vdots \\   0 \end{bmatrix}$$

\begin{example}
The $4 \times 4$ identity matrix is

\[ I  = \begin{bmatrix} 1 & 0 &  0 &   0  \\ 0 & 1 &  0 &   0  \\ 0 & 0 &  1 &   0  \\ 0 & 0 &  0 &  1  \end{bmatrix} \]
 \end{example}
 
 \end{enumerate}

\rule[0.01in]{\textwidth}{0.0025in}
% ---------------------------------------------------- % 

 

\subsection*{Matrix Multiplication Methods}
When performing $A_{m \times n}B_{n \times p} = C_{m \times p}$, 
\begin{enumerate}
	% 1
	\item $AB=C$, then the $c_{ij}$ entry of $AB$ is (row $i$ of $A$) $\cdot$ (column $j$ of $B$)
	
	\[  c_{ij}  = \sum_{k}^n a_{ik} b_{kj}   \;\;\;\; \text{ for each } i \;\; \text{ and } \;\; j \]   
	
	% 2
	\item Perform outer products of the columns of $A$ with the rows of $B$, then add.  That is, 
	\[ AB = (\text{col 1 of }A)(\text{row 1 of  }B) +  (\text{col 2 of }A)(\text{row 2 of  }B) + \cdots + (\text{col n of }A)(\text{row n of  }B) \]



	% 3 
	\item Matrix $A$ times each column of $B$, (form of Block matrices, see below). The columns of $C$ are $A {\bf b_j}$, that is, the $j$th column of $C$ is $A {\bf b_j}$.  In totality, 
	\[  A B = A \begin{bmatrix} {\bf b_1}   &  {\bf b_2}   &   {\bf b_3} &  \dots & {\bf b_p} \end{bmatrix}  =\begin{bmatrix} A{\bf b_1} &   A{\bf b_2}   &  A{\bf b_3} &  \dots & A{\bf b_p} \end{bmatrix} \]
	
	% 4
	\item Every row of $A$ times matrix $B$,  (see Block matrices below)
	
	\end{enumerate}







 
%\rule[0.01in]{\textwidth}{0.0025in}
% ---------------------------------------------------- % 



\subsection*{Block Matrices / Partitions}



\begin{example} Partition a matrix using horizontal and vertical lines.  The example below uses only one horizontal and one vertical.
\[ \begin{bmatrix}[cc|c]  
  	1  &   1 &    2 \\
     	3  &    2  &    1\\
	\hline
     	0  &    1   &  3
     \end{bmatrix}  \]	
\end{example}
 
\begin{example} Partitioned matrix into nine blocks:
\[ A = \begin{bmatrix}[cc|cc|c]  
  	1  &   1 &    2 & 4 & 5 \\
     	3  &    2  &    1 & 3 & 1 \\
	\hline
     	0  &    1   &  3 & 7 & 2 \\ 
	2  &    1   &  1 & 1 & 0 \\ 
	\hline
	1  &   3   &  0 & 1 & 1
     \end{bmatrix}  = \begin{bmatrix} A_{11} &  A_{12} &  A_{13} \\  A_{21} &  A_{22} &  A_{23} \\  A_{31} &  A_{32} &  A_{33} \end{bmatrix} \]	
\end{example}


  

\rule[0.01in]{\textwidth}{0.0025in}
% ---------------------------------------------------- % 






\subsubsection*{Block Multiplication}
Let $A$ be an $m \times n$ matrix and $B$ an $n \times p$. 

\begin{enumerate}
	\item If $B = \begin{bmatrix} B_1 &	B_2 \end{bmatrix}$ where $B_1$ is an $n \times k$ matrix and $B_2$ is an $n \times (p-k)$ matrix, then
	\begin{align*}
	 AB 	&= A ({\bf b}_1, \dots, {\bf b}_k, {\bf b}_{k+1}, \dots, {\bf b}_p) \\
	 	&= (A{\bf b}_1, \dots, A{\bf b}_k, A{\bf b}_{k+1}, \dots, A{\bf b}_p)\\
		&= (A({\bf b}_1, \dots, {\bf b}_k), A({\bf b}_{k+1}, \dots, {\bf b}_p))\\
		&= \begin{bmatrix} AB_1	& 	AB_2 \end{bmatrix}
	\end{align*}
	
	Thus, the matrix $A$ acts like a scalar,
	
	\begin{tcolorbox}[colback=yellow!10!,colframe=gray!15!]
	\[
	AB = A \begin{bmatrix} B_1	& 	B_2 \end{bmatrix} =  \begin{bmatrix} AB_1	& 	AB_2 \end{bmatrix} 
	\]
	\end{tcolorbox}
	
	
\rule[0.01in]{\textwidth}{0.0025in}
% ---------------------------------------------------- % 

	
	% case 2
	\item Let $A_1$ be a $k \times n$ matrix and $A_2$ is an $(m - k) \times n$ matrix.  If $A = \begin{bmatrix} A_1 \\ A_2 \end{bmatrix}$
	then 
	
 	\[ AB = \begin{bmatrix} A_1 \\ A_2 \end{bmatrix} B = \begin{bmatrix} \vec{{\bf a}}_1  \\ \vdots   \\ \vec{{\bf a}}_k \\ \hline \vec{{\bf a}}_{k+1} \\ \vdots \\  \vec{{\bf a}}_m  \end{bmatrix} B = \begin{bmatrix} \vec{{\bf a}}_1 B  \\ \vdots   \\ \vec{{\bf a}}_k B \\ \hline \vec{{\bf a}}_{k+1} B \\ \vdots \\  \vec{{\bf a}}_m B  \end{bmatrix} =  \begin{bmatrix}  \begin{bmatrix} \vec{{\bf a}}_1   \\ \vdots   \\ \vec{{\bf a}}_k  \end{bmatrix}  B  \\  \\  \begin{bmatrix} \vec{{\bf a}}_{k+1}  \\ \vdots \\  \vec{{\bf a}}_m  \end{bmatrix}  B \end{bmatrix}  = \begin{bmatrix} A_1 B \\ A_2 B \end{bmatrix} \]
 	
	
	
	
	
	\begin{tcolorbox}[colback=yellow!10!,colframe=gray!15!]
	\[ AB = \begin{bmatrix} A_1 \\ A_2 \end{bmatrix} B = \begin{bmatrix} A_1 B \\ A_2 B \end{bmatrix} \]
	\end{tcolorbox}
	
	
\rule[0.01in]{\textwidth}{0.0025in}
% ---------------------------------------------------- % 

	
	\item Let $A = \begin{bmatrix} A_1 & A_2 \end{bmatrix}$ where $A_1$ be  a $m \times r$ matrix,  $A_2$ is an $m \times (n - r)$ matrix, and $B = \begin{bmatrix} B_1 \\ B_2 \end{bmatrix}$ matrix with $B_1$ an $r \times q$ matrix and $B_2$ a $(n - r) \times q$ matrix.  Then,     
	
		\begin{tcolorbox}[colback=yellow!10!,colframe=gray!15!]
$$AB = \begin{bmatrix} A_1 & A_2 \end{bmatrix} \begin{bmatrix} B_1 \\ B_2 \end{bmatrix} = A_1B_1 +  A_2B_2$$
	\end{tcolorbox}



\rule[0.01in]{\textwidth}{0.0025in}
% ---------------------------------------------------- % 





\item In general, if the blocks have the proper dimensions, then block multiplication can be carried out in the same manner as ordinary  matrix multiplication.  In particular, if
\[ A =  \begin{bmatrix}  A_{11} & \dots & A_{1n}\\ \vdots & & \vdots \\ A_{m1} & \dots & A_{mn} \end{bmatrix}  \text{\;\;\; and \;\;\;} B =  \begin{bmatrix}  B_{11} & \dots & B_{1p}\\ \vdots & & \vdots \\ B_{n1} & \dots & B_{np} \end{bmatrix}  \]
then
		\begin{tcolorbox}[colback=yellow!10!,colframe=gray!15!]

\[ AB =  \begin{bmatrix}  C_{11} & \dots & C_{1p}\\ \vdots & & \vdots \\ C_{m1} & \dots & C_{mp} \end{bmatrix} \]
where
\[   C_{ij} = \sum_{k=1}^n  A_{ik}B_{kj} \]
	\end{tcolorbox}



\rule[0.01in]{\textwidth}{0.0025in}
% ---------------------------------------------------- % 


\item $A B = A \begin{bmatrix} {\bf b}_1 & \dots & {\bf b}_k \end{bmatrix} = (A{\bf b}_1, A{\bf b}_2, \dots, A {\bf b}_k)$.




%\rule[0.01in]{\textwidth}{0.0025in}
% ---------------------------------------------------- % 


\end{enumerate}
 
 
 

\rule[0.01in]{\textwidth}{0.0025in}
% ---------------------------------------------------- % 






































\subsubsection*{Next time...}
Section 2.5: Inverse Matrices





\subsubsection*{Homework}
\textsection2.4: \#1, 3, 5, 18, 35









%\begin{tcolorbox}[colback=yellow!10!,colframe=gray!15!]
%\begin{theorem}[Cauchy-Schwarz Inequality]
%If ${\bf x} $ and ${\bf y} $ are vectors (in an inner product space) then
 %\[ |  {\bf x} \cdot {\bf y} | \le ||{\bf x}|| \,  ||{\bf y}|| \]
 %\end{theorem}	 
%\end{tcolorbox} 



